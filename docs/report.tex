\documentclass{article}
\usepackage{graphicx}
\usepackage{listings}
\usepackage{hyperref}
\usepackage{color}
\usepackage{float}

\definecolor{dkgreen}{rgb}{0,0.6,0}
\definecolor{gray}{rgb}{0.5,0.5,0.5}
\definecolor{mauve}{rgb}{0.58,0,0.82}

\lstset{frame=tb,
  language=Java,
  aboveskip=3mm,
  belowskip=3mm,
  showstringspaces=false,
  columns=flexible,
  basicstyle={\small\ttfamily},
  numbers=none,
  numberstyle=\tiny\color{gray},
  keywordstyle=\color{blue},
  commentstyle=\color{dkgreen},
  stringstyle=\color{mauve},
  breaklines=true,
  breakatwhitespace=true,
  tabsize=3
}

\title{Milo Mail System Project Report}
\author{Team Milo}
\date{\today}

\begin{document}

\maketitle

\section{Introduction}
This report documents the design and implementation of the Milo Mail System, a scalable web-based email application. The system is built using \textbf{Spring Boot (Java 21/25)} for the backend and \textbf{Angular} for the frontend, using a \textbf{PostgreSQL} database for persistence. The application conforms to Object-Oriented Programming principles and utilizes several GoF design patterns to ensure maintainability and extensibility.

\section{Steps to Run the Code}
\subsection{Prerequisites}
\begin{itemize}
    \item Java Development Kit (JDK) 21 or 25
    \item Maven
    \item Node.js (v18+) and npm
    \item Angular CLI
    \item PostgreSQL Database (running on default port 5432)
\end{itemize}

\subsection{Database Setup}
\begin{enumerate}
    \item Ensure PostgreSQL is running on port \texttt{5432}.
    \item Create a database named \texttt{dilo}.
    \item Update credentials in \texttt{Milo-Backend/src/main/resources/application.properties} if they differ from:
    \begin{verbatim}
    spring.datasource.username=postgres
    spring.datasource.password=12345678
    \end{verbatim}
\end{enumerate}

\subsection{Backend Setup}
\begin{enumerate}
    \item Navigate to the \texttt{Milo-Backend} directory.
    \item Run the following command to start the server:
    \begin{verbatim}
    mvn spring-boot:run
    \end{verbatim}
    \item The backend API will start at \texttt{http://localhost:8080}.
\end{enumerate}

\subsection{Frontend Setup}
\begin{enumerate}
    \item Navigate to the \texttt{Milo} directory.
    \item Install dependencies:
    \begin{verbatim}
    npm install
    \end{verbatim}
    \item Start the development server:
    \begin{verbatim}
    ng serve
    \end{verbatim}
    \item Access the application in your browser at \texttt{http://localhost:4200}.
\end{enumerate}

\section{UML Diagrams}
\subsection{Class Diagram}
The Class Diagram illustrates the structural relationship between the core entities (\texttt{Mail}, \texttt{Attachment}, \texttt{User}) and the behavioral patterns (\texttt{Strategy} hierarchy for sorting, \texttt{Criteria} hierarchy for filtering).

\textit{(Please insert the generated class\_diagram.puml visualization here)}

\subsection{Sequence Diagram}
The Sequence Diagram depicts the workflow of a user filtering and sorting emails. It shows the interaction flow from the Frontend to the Controller, Service, Filter Factory, and Sorting Worker.

\textit{(Please insert the generated sequence\_diagram.puml visualization here)}

\section{Design Patterns Applied}

\subsection{Strategy Pattern (Behavioral)}
\textbf{Usage:} Used for sorting emails dynamically.
\newline
\textbf{Implementation:}
\begin{itemize}
    \item \textbf{Context:} \texttt{SortWorker} classes.
    \item \textbf{Interface:} \texttt{MailSortingStrategy}.
    \item \textbf{Concrete Strategies:} \texttt{SortByDate}, \texttt{SortBySubject}, \texttt{SortByPriority}, etc.
\end{itemize}
This allows the sorting algorithm to be switched at runtime without modifying the client code in the Service.

\subsection{Criteria / Filter Pattern (Structural)}
\textbf{Usage:} Used to filter emails based on multiple conditions (Sender, Subject, Date, etc.).
\newline
\textbf{Implementation:}
\begin{itemize}
    \item \textbf{Interface:} \texttt{Criteria} (defines \texttt{meetCriteria()}).
    \item \textbf{Concrete Criteria:} \texttt{CriteriaSender}, \texttt{CriteriaSubject}, etc.
\end{itemize}
This pattern decouples the filtering logic from the Mail entity and the Service.

\subsection{Factory Pattern (Creational)}
\textbf{Usage:} Used to instantiate the correct Filter/Criteria object based on user input.
\newline
\textbf{Implementation:}
\begin{itemize}
    \item \texttt{CriteriaFactory}: Takes a string (e.g., "subject") and returns a \texttt{CriteriaSubject} instance.
\end{itemize}
This encapsulates object creation logic, simplifying the Service layer.

\subsection{Builder Pattern (Creational)}
\textbf{Usage:} Used for constructing complex \texttt{Mail} objects.
\newline
\textbf{Implementation:} annotated with \texttt{@Builder} (Lombok) in the \texttt{Mail} class. This provides a fluent API for object construction (\texttt{Mail.builder().subject(...).build()}), making the code more readable and reducing constructor complexity.

\subsection{Prototype Pattern (Creational)}
\textbf{Usage:} Used for deep-copying drafts when they are sent.
\newline
\textbf{Implementation:} The \texttt{Mail} class includes a Copy Constructor (\texttt{public Mail(Mail source)}) that creates a deep copy of the mail and its attachments, ensuring that the new "Sent" mail is a distinct entity from the original "Draft".

\section{Design Decisions}
\begin{enumerate}
    \item \textbf{Framework Choice:} Spring Boot was chosen for its robust dependency injection and ease of REST API creation, enabling a clean Microservices-ready architecture.
    \item \textbf{Database:} PostgreSQL was selected for its reliability and advanced SQL features, running on port 5432.
    \item \textbf{Handling Concurrency:} To prevent issues where a user edits a draft while it is being autosaved, we implemented specific backend logic to always treat sent mails as new entities (via Prototype pattern), avoiding Optimistic Locking exceptions.
    \item \textbf{Asynchronous UI:} The Frontend uses Angular Signals and Observables to handle asynchronous file uploads (for large attachments), providing a non-blocking UI with a global loading overlay for better user experience.
\end{enumerate}

\section{User Guide}
\subsection{Registration & Login}
Users can sign up with a unique email. The system checks against the database and alerts if the email is taken.
\begin{figure}[H]
    \centering
    \fbox{\parbox{10cm}{\centering \vspace{2cm} [Insert Login/Signup Screenshot] \vspace{2cm}}}
    \caption{Login Page}
\end{figure}

\subsection{Inbox & Filtering}
Users can view their inbox and apply filters (e.g., by Sender or Subject) and sort the results.
\begin{figure}[H]
    \centering
    \fbox{\parbox{10cm}{\centering \vspace{2cm} [Insert Inbox Screenshot] \vspace{2cm}}}
    \caption{Inbox with Filtering functionality}
\end{figure}

\subsection{Composing Emails}
The Compose window allows sending emails with priority and attachments.
\begin{figure}[H]
    \centering
    \fbox{\parbox{10cm}{\centering \vspace{2cm} [Insert Compose Window Screenshot] \vspace{2cm}}}
    \caption{Compose Email Modal}
\end{figure}

\end{document}
